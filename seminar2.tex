\documentclass[a4paper,10pt]{article}
\usepackage[utf8]{inputenc}
\usepackage{graphicx}

\title{BSIP: Separating uterine EMG records using sample entropy }
\author{Neža Belej (63120340)}

\begin{document}

\maketitle
\section {V podanem tekstu identificirajte gradnike ER modela: atribute, entitetne tipe, identifikatorje, razmerja. Razmislite in opišite ali je potrebno dodati še kakšne druge atribute. }
\begin{itemize}
    \item {Entitetni tipi: } Pacient, obravnava, oddelek, diagnoza, preiskava, predmet preiskave.
    \item{Atributi: } 
    \begin{enumerate}
         \item{Pacient: } Id pacienta, KZZ, ime, priimek, datum rojstva, spol, naslov, pošta.
         \item{Obravnava: } Id obravnave, začetek obravnave, konec obravnave
         \item{Oddelek: } Id oddelka, naziv oddelka
         \item{Diagnoza: } Id diagnoze, naziv diagnoze, utemeljitev, čas diagnoze, ICD USA, ICD WHO, ICD SI
         \item{Preiskava: } Id preiskave, začetek preiskave, konec preiskave, rezultat preiskave
         \item{Predmet preiskave: } Šifra predmeta preiskave, naziv predmeta preiskave, dopustni interval vrednosti (min in max), referenčne vrednosti za ženski spol (min in max), referenčne vrednosti za moški spol (min in max)
    \end{enumerate}
     \item{Identifikatorji }
        \begin{enumerate}
             \item{Pacient: } ID  \\
             Obrazložitev: Zaradi želje po neodvisnosti identifikacije od logike zdravstvenega sistema je identifikator specifični atribut ID. Zakaj ne KZZ ? Zaradi tujcev in neznanih ljudi, sprejetih v obravnavo. Drugi razlog pa je, da se šifriranje KZZ lahko spremeni v prihodnosti. \\
             \item{Obravnava: } ID obravnave, ID pacienta \\
             \item{Oddelek: } ID oddelka \\
             \item{Diagnoza: } ID diagnoze, ID obravnave \\
             Obrazložitev: v moji prvotni rešitvi je bila diagnoza močni entitetni tip, obravnava pa je vsebovala ID diagnoze (na tak način sem shranila diagnozo posameznega pacienta). Ampak ker lahko pacient v okviru ene obravnave dobi več diagnoz (saj pacienta ena obravnava spremlja skozi celoten proces zdravljenja), sem se odločila za zgoraj navedeno identifikacijo. \\
             \item{Preiskava: } ID preiskave, ID obravnave, šifra predmeta preiskave \\
           Obrazložitev: Ker je lahko pacient hkrati na večih obravnavah (zlom noge in rakavo obolenje), zaradi lažje sledljivosti preiskavo (in diagnozo) vežemo z obravnavo in ne s pacientom. Zaradi natančne določitve, za kakšen tip preiskave gre, je pomembna tudi šifra predmeta preiskave. \\
             \item{Predmet preiskave: } Šifra predmeta preiskave
         \end{enumerate}
     \item{Razmerja: } 
        \begin{enumerate}
             \item{Pacient - obravnava (1,n)}
             \item{Obravnava - oddelek (n,n)}
             \item{Obravnava - diagnoza (1,n)}
              \item{Obravnava - preiskava (1,n)}
              \item{Preiskava - predmet preiskave (n,1)}
         \end{enumerate}
\end{itemize}  

\end{document}


